\documentclass{article}
\usepackage[utf8]{inputenc}
\renewcommand{\contentsname}{Contenuti}

\title{Tirocini curriculari}
\author{Massimo Albino Sandretti}
\date{Anno Scolastico 2020-2021}

\begin{document}

\maketitle


\tableofcontents
\newpage

\section{Introduzione}
\subsection{Realtà d'interesse}
Si deve realizzare un sito web dinamico in grado di gestire i dati relativi ai tirocini formativi degli studenti.
Il progetto consiste nella realizzazione di una banca dati che modelli i dati coinvolti e nella implementazione
di un sito web dinamico che permetta l'utilizzo del data base tramite adeguata interfaccia web.
Un Istituto di Istruzione Superiore vuole gestire i dati relativi ai Percorsi per le Competenze Trasversali e per
l’Orientamento (PCTO) degli allievi.
Occorre registrare i dati relativi alle aziende, ai docenti tutor e agli allievi, suddivisi per classe e per indirizzo.
Al sito possono accedere i docenti tutor e le aziende, che vengono identificati dopo una fase di autenticazione.
Le aziende accedono per scaricare e stampare la documentazione relativa ai contratti di collaborazione e per
conoscere i dati degli alunni che svolgeranno l'attività di tutoraggio presso di loro, mentre i docenti tutor
possono stampare le convenzioni con le aziende e inserire i dati della valutazione aziendale al termine del
PCTO.
Il database da realizzare deve contenere i dati utili delle aziende convenzionate che hanno dato la propria
adesione e quelli degli allievi con gli abbinamenti allievo - azienda, allievo - docenti tutor e allievo - tutor
aziendale.
Il sito web deve mostrare diverse interfacce una per ogni categoria di utenti; l'interfaccia della pagina web deve
essere una sorta di "cruscotto elettronico" nel quale l'utente può decidere l'operazione da svolgere.
Il candidato analizzi la realtà di riferimento e, fatte le opportune ipotesi aggiuntive, individui una soluzione
che a suo motivato giudizio sia la più idonea per la gestione della base di dati.
\subsection{Analisi}
\subsection{Ipotesi aggiuntive}
\begin{itemize}
    \item L'amministratore inserisce i dati degli studenti e dei docenti
    \item Una volta inseriti i dati, sarà generata una chiave d'accesso che poi sarà inviata via email all'azienda, al tutor, allo studente o al docente per permettergli di accedere e visionare/inserire/modificare i propri dati.
    \item L'amministratore effettua il collegamento studente-tutor. Inoltre può anche eliminare o modificare tale collegamento
    \item Il responsabile d'indirizzo può visionare i dati degli studenti facenti parte degli indirizzi di cui è responsabile e le relative associazioni con i tutor e le aziende.
    \item Il tutor (aziendale) è una figura nominata dalla azienda, che segue gli studenti durante il PCTO. Un tutor può seguire più studenti (anche contemporaneamente).
    \item Dato che l'attività di PCTO viene generalmente svolta più di una volta, uno studente potrebbe anche avere più tutor.
    \item L'azienda che accede può visionare i dati degli studenti che svolgeranno o hanno svolto il PCTO presso di loro
    \item L'azienda inoltre può inserire e modificare i dati dei propri tutor. Tuttavia dopo che ha inserito i dati, per confermare la registrazione, serve l'autorizzazione di un amministratore
    \item Il tutor può visionare i dati degli studenti che svolgeranno PCTO, nella azienda che rapprsenta, sotto la sua supervisione.
    \item Il responsabile può effettuare la valutazione aziendale al termine del PCTO
    \item Nulla vieta che uno studente possa effetturare il PCTO nella stessa azienda sotto lo stesso tutor.
    \item Un indirizzo non può avere più responsabili, ma un responsabile può essere attribuito a più indirizzi
    \item Un indirizzo può avere più classi, ma ogni classe può avere un solo indirizzo
    \item Un amministratore è un responsabile particolare che non può essere collegato ad un indirizzo
    
\end{itemize}
\subsection{Glossario}

\section{Progettazione del Database}
\subsection{Progettazione concettuale}
\subsubsection{Entità-attributi}
\begin{itemize}
    \item Indirizzi
    \begin{itemize}
        \item sigla
        \item nome
    \end{itemize}
    \item Classi
    \begin{itemize}
        \item id\_classe
        \item anno
        \item sezione
    \end{itemize}
    \item Responsabili
    \begin{itemize}
        \item id\_responsabile
        \item nome
        \item cognome
        \item password
        \item email
    \end{itemize}
    \item Studenti
    \begin{itemize}
        \item id\_studente
        \item nome
        \item cognome
        \item password
        \item email
    \end{itemize}
    \item Aziende
    \begin{itemize}
        \item id\_azienda
        \item denominazione
        \item password
        \item email
    \end{itemize}
    \item Tutor
    \begin{itemize}
        \item id\_tutor
        \item nome
        \item cognome
        \item password
        \item email
        \item autorizzato
    \end{itemize}
    
\end{itemize}
\subsubsection{Associazioni}
\begin{itemize}
    \item Rappresenta(Tutor, Aziende)\\*
    1 Tutor rappresenta 1 Azienda\\*
    1 Azienda è rappresentata da N tutor
    \item Segue$_1$(Tutor, Studenti)\\*
    1 Tutor segue N Studenti\\*
    1 Studente è seguito da N Tutor
    \item Segue$_2$(Responsabili, Indirizzi)\\*
    1 Responsabile segue N Indirizzi\\*
    1 Indirizzo è seguito da 1 Responsabili
    \item Riguarda(Classi, Indirizzi)\\*
    1 Classe riguarda 1 Indirizzo\\*
    1 Indirizzo è riguardato da N Classi
    \item Appartiene(Studenti, Classi)\\*
    1 Studente appartiene ad 1 Classe\\*
    ad 1 Classe appartengono N Studenti
    
\end{itemize}

\subsubsection{Schema ER grezzo}
\subsubsection{Ristrutturazione schema ER}
\subsubsection{Schema ER ristrutturato}

\subsection{Progettazione logica}
\subsubsection{Schema logico}
\subsubsection{Verifica normalizzazione}
\subsubsection{Tracciato record}

\subsection{Progettazione fisica}
\subsubsection{Creazione tabelle MySQL}
\subsubsection{Implementazione vincoli}

\section{Interfaccia Web}
\subsection{Installazione Database}
\subsection{Utenti}
\subsubsection{Amministratore}
\subsubsection{Aziende}
\subsubsection{Docenti}

\subsection{Pagine}
\subsubsection{Amministrazione}
\subsubsection{Aziende}
\subsubsection{Docenti}

\section{Sicurezza dei dati????}

\end{document}
